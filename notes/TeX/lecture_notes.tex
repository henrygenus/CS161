\documentclass[9pt]{extarticle}
\usepackage[T1]{fontenc}
\usepackage[textheight=10in]{geometry}
\usepackage[english]{babel}
\usepackage{enumitem}
\usepackage[normalem]{ulem}
\usepackage{amsthm} 
\usepackage{amssymb}
\usepackage{amsmath} 
\usepackage{graphicx}
\usepackage{wrapfig}
\graphicspath{ {Images/} }
\usepackage{caption}
\usepackage{floatrow}
\usepackage[edges]{forest}
\usepackage{tikz-qtree}
\usepackage{float}
\usepackage{multicol}
\usepackage{xcolor}
\usepackage{listings}
\lstset{basicstyle=\ttfamily,
	showstringspaces=false,
	commentstyle=\color{red},
	keywordstyle=\color{blue}
}
\usepackage{tikz}
\usetikzlibrary{positioning, calc, shapes.geometric, shapes.multipart, 
	shapes, arrows.meta, arrows, 
	decorations.markings, external, trees}
\tikzset{
	treenode/.style = 	{shape=rectangle, rounded corners,
					draw, align=center,
					top color=white, bottom color=blue!30},
	no/.style = 	{treenode, bottom color=red!30},
	yes/.style = 	{treenode, bottom color=green!30},
	env/.style = 	{treenode, font=\ttfamily\normalsize},
}
\usepackage{blindtext}
\usepackage{subfiles}
\tolerance=1
\emergencystretch=\maxdimen
\hyphenpenalty=10000
\hbadness=10000
\setlength{\itemsep}{0em}

\title{%
	CS 161: Artificial Intelligence\\
	Lecture Notes}
\author{Henry Genus}
\date{Spring 2020}

\begin{document}
\maketitle
\tableofcontents
\newpage

\section{Introduction}
\subfile{Lectures/Lecture_1/Introduction}
\newpage

\section{LISP}
\subfile{Lectures/Lecture_2/LISP}
\newpage

\section{Problem Solving as Search}
\subfile{Lectures/Lecture_3/Problem_Solving_as_Search}
\newpage

\section{Blind Search Strategies}
\subfile{Lectures/Lecture_4/Blind_Search_Strategies}
\newpage

\section{Heuristic Search}
\subfile{Lectures/Lecture_5/Heuristic_Search}
\newpage

\section{Constraint Satisfaction}
\subfile{Lectures/Lecture_6/Constraint_Satisfaction}
\newpage

\section{Two Player Games}
\subfile{Lectures/Lecture_7/Two_Player_Games}
\newpage

\section{Knowledge Representation}
\subfile{Lectures/Lecture_8/Knowledge_Representation}
\newpage

\section{Classical Propositional Logic}
\subfile{Lectures/Lecture_9/Classical_Propositional_Logic}
\newpage

\section{Modern Propositional Logic}
\subfile{Lectures/Lecture_10/Modern_Propositional_Logic}
\newpage

\section{First Order Logic Representation}
\subfile{Lectures/Lecture_11/First_Order_Logic_Representation}
\newpage

\section{First Order Logic Inference}
\subfile{Lectures/Lecture_12/First_Order_Logic_Inference}
\newpage

\section{Probabilistic Reasoning}
\subfile{Lectures/Lecture_13/Probabilistic_Reasoning}
\newpage

\section{Bayesian Networks}
\subfile{Lectures/Lecture_14/Bayesian_Networks}
\newpage

\section{Modeling and Inference}
\subfile{Lectures/Lecture_15/Modeling_and_Inference}
\newpage

\section{Bayesian Learning}
\subfile{Lectures/Lecture_16/Bayesian_Learning}
\newpage

\section{Decision Trees and Random Forests}
\subfile{Lectures/Lecture_17/Decision_Trees_and_Random_Forests}
\newpage

\section{Neural Networks}
\subfile{Lectures/Lecture_18/Neural_Networks}
\newpage

\section{Neural Network Applications}
\subfile{Lectures/Lecture_19/Neural_Network_Applications}
\newpage

\end{document}