
\documentclass[../../lecture_notes.tex]{subfiles}

\begin{document}

\noindent The tool we will use to work with probabilistic reasoning is called a \textbf{\underline{Bayesian Network}}.\\
This consists of two major components:
\begin{enumerate} [itemsep=0mm]
	\item Directed Acyclic Graph shows casualty
	\item Numbers/Nodes represent probability
\end{enumerate}
\noindent Notation:
\begin{itemize} [itemsep=0mm]
	\item Variable — X
	\item Value — x
	\item Variable Probability Distribution — Pr(x) = Pr(X=x)
	\item Set of Variables — X
	\item Instantiation — x
	\item World Probability Distribution — Pr(X) = truth table
\end{itemize} \medskip

\noindent Remember that if $Pr(\alpha|\beta, \gamma) = Pr(\alpha|\gamma)$, 
	then $\alpha\ \&\ \beta$ are independent given $\gamma$\\
We can apply the same idea to sets of variables:\\
	\indent X and Y are independent given Z if Pr(x|y, z) = Pr(x|z)\\
	\indent We write this as I(X, Z, Y)\\
\\
Consider the following:
\begin{center}\begin{tikzpicture}
	\node[circle, draw, align=center] (E) {E};
	\node[circle, draw, below left =of E, align=center] (R) {R};
	\node[circle, draw, below right =of E, align=center] (A) {A};
	\node[circle, draw, below =of A, align=center] (C) {C};
	\node[circle, draw, above right =of A, align=center] (B) {B};
	\draw [->] (E.south west) -- (R.north east);
	\draw [->] (E.south east) -- (A.north west);
	\draw [->] (B.south) -- (A.north east);
	\draw [->] (A.south) -- (C.north);
\end{tikzpicture}\end{center}

\subsubsection*{Logic}
\noindent The alarm is triggered by burglary or earthquake.\\
A radio report is caused by an earthquake.\\
A call from the neighbor may come post-alarm.

\subsubsection*{Independence}
\noindent R \& C are independent given A — I(R, A, C)\\
E \& B are independent — I(E, $\phi$, B)\\
\\
This can be much more efficient than a table.\\
The independence and parameters uniquely identify the graph.\\
\\
We introduce a few terms:
\begin{itemize} [itemsep=0mm]
	\item Parents(V) — the nodes pointing through V via an edge\\
		Parents(A) = \{E, B\} \& Parents(R) = \{E\}
	\item Descendants(V) — the nodes reachable by V\\
           	Descendants(E) = \{R, A, C\} \& Descendants(B) = \{A, C\} \& Descendants(X) = \{\}
	\item Non-Descendants(V) — the nodes unreachable from V, excluding the self \& parents\\
		Non-descendants(A) = \{R\} \& Non-descendants(E) = \{B\}
\end{itemize}

\noindent We call these independence statements the \textbf{\underline{Marcovion Assumptions}}\\
	\indent $\equiv$ all the defined independence statements of a Bayesian Network\\
	\indent These are of the form I(V, Parents(V), Non-Descendants(V))\\
Marcov(G) = \{ I(C, A, BE), I(R, E, ABC), I(A, BE, R), I(B, $\phi$, ER), I(E, $\phi$, B) \}\\
\\
We can parametrize the structure by assigning weight to the connections.\\

\begin{center}\begin{tikzpicture}
	\node [ellipse, draw, align=center] (A) {Winter?\\(A)};
	\node [ellipse, draw, align=center, below left =of A] (B) {Sprinkler?\\(B)};
	\node [ellipse, draw, align=center, below right =of A] (C) {Rain?\\(C)};
	\node [ellipse, draw, align=center, below right =of B] (D) {Wet Grass?\\(D)};
	\node [ellipse, draw, align=center, below right =of C] (E) {Slippery Road?\\(E)};
	\draw [->] (A.south west) -- (B.north east);
	\draw [->] (A.south east) -- (C.north west);
	\draw [->] (B.south east) -- (D.north west);
	\draw [->] (C.south west) -- (D.north east);
	\draw [->] (C.south east) -- (E.north west);
\end{tikzpicture}\end{center}\begin{center}
\begin{tabular} { | c | c | } \hline A & $\theta_A$ \\\hline 1 & 0.6 \\ 0 & 0.4 \\\hline\end{tabular}
\begin{tabular} { | c c | c | } \hline
	A & B & $\theta_{B|A}$ \\\hline 
	1 & 1 & 0.2 \\
	1 & 0 & 0.8 \\
	0 & 1 & 0.75 \\
	0 & 0 & 0.25
\\\hline\end{tabular}
\begin{tabular} { | c c | c | } \hline
	A & C & $\theta_{C|A}$ \\\hline 
	1 & 1 & 0.8 \\
	1 & 0 & 0.2 \\
	0 & 1 & 0.1 \\
	0 & 0 & 0.9
\\\hline\end{tabular}
\begin{tabular} { | c c c | c | } \hline
	B & C & D & $\theta_{D|B,C}$ \\\hline 
	1 & 1 & 1 & 0.95 \\
	1 & 1 & 0 & 0.05 \\
	1 & 0 & 1 & 0.9 \\
	1 & 0 & 0 & 0.1 \\
	0 & 1 & 1 & 0.8 \\
	0 & 1 & 0 & 0.2 \\
	0 & 0 & 1 & 0 \\
	0 & 0 & 0 & 1
\\\hline\end{tabular}
\begin{tabular} { | c c | c | } \hline
	C & E & $\theta_{E|C}$ \\\hline 
	1 & 1 & 0.7 \\
	1 & 0 & 0.3 \\
	0 & 1 & 0 \\
	0 & 0 & 1
\\\hline\end{tabular}\end{center}

\noindent Note that $Pr(\neg d, b, \neg c) = \theta_\{\neg d | b, \neg c\}$\\
\\
Consider an arbitrary row abcde \\ 
\indent $Pr(abcde) =\theta_a\ \theta_{b|a}\ \theta_{c|a}\ \theta_{d|bc}\ \theta_{e|c} = 0.7$\\
\indent $Pr(ab\neg cd\neg e) =\theta_{a}\ \theta{b|a}\ \theta_{\neg c|a}\ \theta_{d|b\neg c}\ \theta_{\neg e|c} = 0.216$\\
\indent $Pr(\neg(abcde)) = \theta_{\neg a}\ \theta_{\neg b|\neg a}\ \theta_{\neg c|\neg a}\
	\theta_{\neg d|\neg b\neg c}\ \theta_{\neg e|\neg c} = 0.09$\\
\\
We can evaluate the independence of any two nodes are independent given a third by
\subsection*{Deseperation}
\noindent This uses an expansion on Markov’s much more efficient than probability theory.\\
If setting Z blocks paths x$\rightarrow$y, x \& y are d-separated given Z\\
There are three ways for a node to be connected in a path, treated as follows:
\begin{itemize} [itemsep=0mm]
	\item sequential $\equiv\ \rightarrow w \rightarrow$ is blocked iff w $\in$ \textbf{Z}
	\item divergent $\equiv\ \leftarrow w \rightarrow$ is blocked iff w $\in$ \textbf{Z}
	\item convergent $\equiv\ \rightarrow w \leftarrow$ is blocked iff w \& Decendants(w) $\notin$ \textbf{Z}
\end{itemize}

\noindent We can clarify by giving an example:
\begin{center}\begin{minipage} {0.6\textwidth}\begin{center} \begin{tikzpicture}
	\node [circle, draw] (A) {A};
	\node [circle, draw, below =of A] (T) {T};
	\node [circle, draw, below right =of T] (P) {P};
	\node [circle, draw, below left =of P] (X) {X};
	\node [below right =0.8 of P] (t) {};
	\node [circle, draw, right =0.4 of t] (D) {D};
	\node [circle, draw, above right =0.8cm of P] (C) {C};
	\node [above right =of C] (tt) {};
	\node [circle, draw, below =0.1cm of tt] (S) {S};
	\node [circle, draw, below right =of S] (B) {B};
	\draw [->] (A.south) -- (T.north);
	\draw [->] (T.south east) -- (P.north west);
	\draw [->] (P.south west) -- (X.north east);
	\draw [->] (P.south east) -- (D.north west);
	\draw [->] (C.south west) -- (P.north east);
	\draw [->] (S.south west) -- (C.north east);
	\draw [->] (S.south east) -- (B.north west);
	\draw [->] (B.south west) -- (D.north east);
\end{tikzpicture}\end{center}\end{minipage}%
\begin{minipage} {0.4\textwidth}
dsep(C, S, B)?\\
$\rightarrow$ open(BSC)? false.\\
$\rightarrow$ open(CPDV)?\\
--$\rightarrow$ open(SPD)? true.\\
--$\rightarrow$ open(PDS)? false.\\
$\implies$ dsep(C, S, B) = true.
\end{minipage} \end{center}
\end{document}