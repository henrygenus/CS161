\documentclass[../../lecture_notes.tex]{subfiles}

\begin{document}

\noindent We will attempt to systemically answer the following question:\\
\indent Does $\Delta$ imply $\alpha$?\\
\\
As well as the related questions:\\
	\indent Is $\Delta$ equivalent to $\alpha$?\\
	\indent Is $\Delta$ satisfiable?\\
	\indent Is $\Delta$ valid?\\
\\
We will discuss four major methods:
\begin{enumerate} [itemsep=0mm]
	\item Enumerating Models
	\item Inference Rules
	\item Search (CSP/SAT)
	\item NNF Circuits
\end{enumerate}
(1) and (2) are deprecated but (2) is fundamental to (3) and (4).\\
This lecture will discuss the first two, and the next two will be discussed in the next.\\

\subsection*{Inference By Enumerating Models}

\begin{center} \begin{minipage}{0.6\textwidth}
\noindent Suppose our knowledge base $\Delta = \{A, A \lor B \implies C\}$.\\
Let $\alpha = C$; we can then ask: Does $\Delta$ imply $\alpha$?
\begin{enumerate} [itemsep=0mm]
	\item convert our knowledge base to CNF\\
		$\Delta = A \land (A \lor B \implies C) = A \land (\neg(A \lor B) \lor C)$
	\item build a truth table for the KB
	\item use the models to solve:\\
		$M(\Delta) = \{W1, W3\}$\\
		$M(\alpha) = \{W1, W3, W5, W7\}$
	\item If $M(\Delta) \subseteq M(\alpha)$, then $\Delta \models \alpha$
\end{enumerate}
\noindent Complexity: $O(2^n)$, n = \# variables \\(it can be used up to 25 variables!)\\
We can just as easily use this to solve the other questions.
\end{minipage}%
\begin{minipage}{0.4\textwidth}
\begin{tabular}{ | c || c | c | c | c | c | }
	\hline
	& A & B & C & $\Delta$ & $\alpha$\\ 
	\hline\hline
	$W_1$ & T & T & T & T & T\\
	\hline
	$W_2$ & T & T & F & F & F\\
	\hline
	$W_3$ & T & F & T & T & T\\
	\hline
	$W_4$ & T & F & F & F & F\\
	\hline
	$W_5$ & F & T & T & F & T\\
	\hline
	$W_6$ & F & T & F & F & F\\
	\hline
	$W_7$ & F & F & T & F & T\\
	\hline
	$W_8$ & F & F & F & F & F\\
	\hline
\end{tabular} \end{minipage} \end{center}

\noindent Many logical systems demonstrate \textbf{\underline{monotonicity}}.\\
\indent $\equiv$ as the information increases, the set of entailed sentences can only grow.\\
For these problems, enumeration can cause the set to explode;\\
\indent we need a model that can ignore irrelevant propositions.

\subsection*{Inference By Resolution}
\noindent This method starts from the \textbf{\underline{Deduction Theorem}}:
\indent Any question of implication can be converted into a SAT question.\\
\\
If we wish to solve a SAT question, we can attempt to prove by contradiction.
\indent Say we want to prove that $\Delta$ implies $\alpha$:
\begin{enumerate} [itemsep=0mm]
	\item assume $\Delta$ \& $\neg \alpha$ 
	\item show that this causes a contradiction
\end{enumerate} \medskip

\noindent  To solve questions of satisfiability, we will use the concept of \textbf{\underline{inference rules}}.\\
An inference rule is of the form $\frac{P_1, P_2} {P_3}$.\\
\indent This says “if P1 and P2, then P3”.\\
\\
To apply proof by contradiction to questions of satisfiability, we:
\begin{enumerate} [itemsep=0mm]
	\item convert the implication to normal form
	\item apply inference rules until we either terminate or have a contradiction
\end{enumerate} \medskip

\noindent If we can use rule R for a given derivation, we use the symbol $\vdash_R$.\\
If a rule will derive $\alpha$ from $\Delta$ any time $\Delta \not\models \alpha$, 
	we call the rule \textbf{\underline{complete}}.\\
\indent \textbf{\underline{Modus Ponens}} $\frac{A, A\implies B}  {B}$ is not complete.\\
If a rule’s denominator is always true provided the numerator, we call the rule \textbf{\underline{sound}}.\\
\indent $\frac {A, (A \lor B)} {(A \land B)}$ is not sound.\\
\\
We would assume a sound and complete rule would be needed for proofs, \\
\indent BUT our primary rule is actually not complete:
\begin{equation*} \text{Resolution: } \ \frac{A \lor B, \neg B \lor C } {A \lor C} \end{equation*}
This may be counterintuitive, but we can see it holds:\\
	\indent $A \lor B = \neg A \implies B$\\
	\indent $\neg B \lor C = B \implies C$\\
	\indent $\neg A \implies B \ \&\&\  B \implies C = \neg A \implies C = A \lor C$\\
This is not complete, but it is \textbf{\underline{refutation complete}} when applied to a CNF.\\
\indent $\equiv$ given any CNF with a contradiction, it will derive the contradiction.\\

\begin{center} \begin{minipage}{0.8\textwidth}
The Algorithm:
We are asked the question "Does $\Delta$ imply $\alpha$?"\\
This is equivalent to "Is $\Delta  \land \neg \alpha$ a contradiction?".\\
We therefore apply resolution until either:
\begin{enumerate} [itemsep=0mm]
	\item a contradiction occurs, in which case $\Delta$ implies $\alpha$
	\item no more resolution can be applied, in which case it does not
\end{enumerate}
Example 1:\\
\indent $\Delta: ((A \lor \neg B) \implies C) \land (C \implies D \lor \neg E) \land (E \lor D)$\\
\indent $\alpha: A \implies D$\\
Does $\Delta$ imply $\alpha$? $\rightarrow$ Is $\Delta \land \neg \alpha$ unsatisfiable?\\
\indent The empty set cannot be true, so $\Delta$ implies $\alpha$\\
Example 2:\\
\indent $\Delta: ((A \land B) \implies C) \land A \land (C \implies D)$\\
\indent $\alpha: C$\\
If we apply resolution, the algorithm will terminate; therefore, $\Delta$ does not imply $\alpha$.
\end{minipage}%
\begin{minipage}{0.2\textwidth} \begin{tikzpicture}
\node [rectangle, draw, text width=\textwidth] {\begin{enumerate} [itemsep=0mm] \setcounter{enumi}{0}
	\item $\neg A \lor C$
	\item $B \lor C$
	\item $\neg C \lor D$
	\item $E \lor D$
	\item $A$
	\item $\neg D$\\
	\item $C <0, 4>$
	\item $D \lor \neg E <2, 6>$
	\item $E <3, 5>$
	\item $D <7, 8>$
	\item $\Phi <5, 9>$
\end{enumerate}};
\end{tikzpicture}\end{minipage}\end{center}

\noindent This algorithm depends on a CNF, but how do we get one?
\begin{enumerate} [itemsep=0mm]
	\item Get rid of all connections, but for $\{\lor \land \neg\}$
		\begin{itemize} [itemsep=0mm]
			\item $A \iff B \rightarrow (A \implies B) \land (B \implies A) $
			\item $A \implies B \rightarrow \neg A \lor B$
		\end{itemize}
	\item Use deMorgan’s Law to push negatives inward
		\begin{itemize} [itemsep=0mm]	
			\item $\neg(A \land B) \rightarrow \neg A \lor \neg B$
			\item $\neg(A \lor B) \rightarrow \neg A \land \neg B$
		\end{itemize}
	\item Distribute $\lor$ over $\land$
		\begin{itemize} [itemsep=0mm]
			\item $(A \land B) \lor C \rightarrow (A \lor C) \land (B \lor C)$
		\end{itemize}
\end{enumerate}

Example: $A \iff (B \lor C)$
\begin{enumerate} [itemsep=0mm]
	\item $\rightarrow (A \implies (B \lor C)) \land ((B \lor C) \implies A)$
                  $\rightarrow (\neg A \lor B \lor C) \land (\neg(B \lor C) \lor A)$
	\item $\rightarrow (A \lor B \lor C) \land ((\neg B \land \neg C) \lor A)$
	\item $\rightarrow (\neg A \lor B \lor C) \land ((\neg B \lor A) \land (\neg C \lor A)$
\end{enumerate}
\noindent Coincidentally, this is an NNF.

\end{document}