\documentclass[../../lecture_notes.tex]{subfiles}

\begin{document}

\begin{itemize} [itemsep=0mm]
	\item is one of the oldest languages (1958)
	\item is the second high level language developed (behind FORTRAN (1957))
	\item was introduced for symbolic manipulation
	\item is a functional language
	\item has a uniform and simple syntax
\end{itemize} \medskip

\noindent Upon starting the LISP IDE, the user sees a listener (>).\\
The user then gives a LISP expression of the form:
\begin{lstlisting} [language=Lisp]
    (op arg1 ... argn) ; where op is a function name or special operator
\end{lstlisting} \medskip

\noindent Types of expressions:
\begin{enumerate}
\item Numeric Expressions
\item Symbolic Expressions
\item Boolean Expressions
\end{enumerate}

\noindent Other Material:
\begin{enumerate}
\item Branching 
\item Function Definition
\end{enumerate}
\noindent and that is LISP!\\
\
Lisp uses prefix notation for operations:
\begin{lstlisting} [language=Lisp]
; Mixing of types in an expression is allowed:
> (+ 2.7 10)
> 12. 7
; There is no limit on the argument count
> (+ 21 35 12 7)
> 75
; Expressions can be recursive
> (+ (* 3 51) (- 10 6))
> 19
\end{lstlisting} \medskip

\noindent We introduce two special operators:
\begin{enumerate} 
\item The quote operator gets an expression without evaluation.\\
	We use this since sometimes lists represent data rather than expressions.
\begin{lstlisting} [language=Lisp]
> (quote (+ 3 1))
> (+ 3 1)
; this can also be written "> '(+ 3 1)"
\end{lstlisting}
\item The setq operator assigns a value to a variable.
\begin{lstlisting} [language=Lisp]
> (setq x 3)
> 3
> (setq y (+ 1 3))
> y
> 4
; we can use this to bind an expression to a variable
> (setq z '(+ 1 3))
> (+ 1 3)
; expressions in expressions are evaluated
> (setq z y)
> 4
\end{lstlisting}
\end{enumerate} \medskip

\noindent More generally, a LISP expression is either
\begin{enumerate}
\item an atom (number, symbol, string, etc)
\item a list (>= 0 expressions)
\end{enumerate} \medskip

\noindent A common way to use lists/expression is to represent data; we then need:
\begin{enumerate}
\item accessors (car, cdr) = (first, rest)
\item constructors (cons, list)
\end{enumerate}
\noindent The accessors car and cdr behave as follows:
\begin{lstlisting} [language=Lisp]
> (setq x '(a b c d)) // this would be an error without the quote
> (a b c d)
; car (or first) gives the first item in a list
> (car x)
> a
; cdr (or rest) gives the rest of the list
> (cdr x)
> (b c d)
; how would I print the second element?
> (car (cdr x)) // shorthand: (cadr x)
> b
; we can recurse indefinitely here
> (caddr x)
> c
\end{lstlisting}
The empty list is denoted “NIL” and evaluates to false.\\
NIL functions expectedly with car/cdr
\begin{lstlisting} [language=Lisp]
> (car NIL)
> NIL
> (cdr NIL)
> NIL
> (cdr '(c))
> NIL
\end{lstlisting}

\noindent The constructors list and cons behave as follows:\\
\begin{lstlisting} [language=Lisp]
; list joins the following elements
> (list 1 2 3)
> (1 2 3)
; cons joins the two parameters, such that arg1=car \& arg2=cdr
> (cons 'a '(b c))
> (a b c)
> (cons (+ 1 2) '(b c))
> (3 b c)
> (cons 'a NIL)
> (a)
> (cons '(a b) '(c d))
> ((a b) c d)
> (cons 1 (cond 2 NIL))
> (1 2)
\end{lstlisting} \medskip

\noindent Boolean expressions use NIL for false and t for true
\begin{lstlisting} [language=Lisp]
> (> 3 1)
> t
> (< 3 1)
> NIL
; listp evaluates type of element
> (not (listp 3))
> t
> (listp '(a b))
> t
; there is an =NIL operator 'null'
> (null (cdr '(2)))
> t
; note that it uses the most recent true value for return
> (OR NIL 3)
> 3
> (OR NIL (cdr '(c)) 7)
> 7
\end{lstlisting} \medskip

\noindent We can evaluate equality in one of three ways:\\
\begin{itemize}
	\item '=' compares integers
	\item 'equal' compares elements (or direct values)
    	\item 'eqL' compares underlying pointers
\end{itemize} \medskip

\noindent Now onto the big stuff……
\subsection*{Functions}
\begin{lstlisting} [language=Lisp]
> (defun name (parameters)
      operations))
\end{lstlisting} \medskip

\noindent A very simple function to start with is the “square” function
\begin{lstlisting} [language=Lisp]
> (defun square (x)
      (* x x))
> (square 3)
> 9
\end{lstlisting} \medskip

\noindent We can use the “cond” keyword to emulate a switch
\begin{lstlisting} [language=Lisp]
> (defun odd (x)
      (cond ((= x 0) NIL)
	((= x 1) t)
	(t (odd (- x 2)))))
\end{lstlisting} \medskip

\noindent In Lisp, we tend to write functions that fall into one of three types:
\begin{enumerate}
	\item numeric (as above)
	\item list
	\begin{enumerate}
		\item use accessors
		\item use constructors
	\end{enumerate}
\end{enumerate} \medskip

\noindent A quick note on numeric functions:
\begin{lstlisting} [language=Lisp]
; we bind variables with the keyword "let
> (defun foo (x y)
      (let ((a (+ x y))
	(b (* x y)))
	(/ a b)))
; this binding is local to the scope of the operation
> (setq x 5)
> (+ (let ((x 3)) (+ x (* x 10))) x)
> 38
; binding is done in parallel
> (setq x 2)
> (let ((a (+ x -1))
      (b (+ a 3)}
; GIVES AN ERROR: let* binds in series \& would allow the above:
> (let ((x 3) (y (+ x 2))) (x + y)
> 12
> (let* ((x 3) (y (+ x 2))) (x + y)
> 15
\end{lstlisting} \medskip

\noindent We consider some accessor functions:
\begin{enumerate}
\item compute the sum of all the elements of a list:
\begin{lstlisting} [language=Lisp]
> (defun sumlist (L)
(cond ((null L) 0)
  (t (+ (first L) (sumlist (rest L)}
; a curly brace denotes "close all paren"
> (sumlist `(1 2 3))
> 6
\end{lstlisting}
\item determine whether an element is a member of a list:
\begin{lstlisting} [language=Lisp]
> (defun member (x L)
      (cond ((null t) NIL)
	((equal x (first L)) t)
	(t (member x (rest L)}
> (member NIL `(a b))
> NIL
> (member `(a b) `(1 7 (a b) d))
> t
\end{lstlisting}
\item return the last element in a list
\begin{lstlisting} [language=Lisp]
> (defun last (L)
      (cond ((null (rest L)) (first L))
	(t (last (rest L)}
\end{lstlisting} \medskip
\end{enumerate} 

\noindent We consider some constructor functions:
\begin{enumerate}
\item remove one occurrence of an element from a list 
\begin{lstlisting} [language=Lisp]
> (defun remove (elm list)
      (cond ((null list) NIL)
	((equal elm (first list)) (rest list)
	(t (cons (first list) (remove elm (rest list)}
\end{lstlisting}
\item append a list to another
\begin{lstlisting} [language=Lisp]
> (defun append (L1 L2)
      (cond ((null (last L2)) L2)
	(t (cons (first L2) (append L1 (rest L2)}
\end{lstlisting}
\item reverse a list
\begin{lstlisting} [language=Lisp]
> (defun reverse (L)
      (cond ((null L) NIL)
	((null (rest L) (first L))
	(t (append (reverse (rest L) (first L)}
\end{lstlisting}
\end{enumerate}

\end{document}