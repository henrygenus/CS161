\documentclass[../../lecture_notes.tex]{subfiles}

\begin{document}

\noindent Search problems work as follows:

\begin{center} \begin{tikzpicture}
	\node[rectangle, draw, align=center] (1) {Search\\Problem};
	\node[rectangle, draw, align=center, right =of 1] (2) {Search\\Engine};
	\node[rectangle, draw, align=center, below =of 2] (3) {Search\\Strategy};
	\node[align=center, right =of 2] (4) {Solution};
	\draw [->] (1.east) -- (2.west);
	\draw [->] (3.north) -- (2.south);
	\draw [->] (2.east) -- (4.west);
\end{tikzpicture}\end{center}
Solving a search problem consists of answering the following two questions:
\begin{enumerate}[itemsep=0mm]
	\item How do you format a problem as a search problem?
	\item What search strategy do you use (uninformed/blind or informed/heuristic)?
\end{enumerate}

Search problems have 3 major parts:
\begin{enumerate}[itemsep=0mm]
\item Initial State
\item Final/Goal State
\item Actions/Operators
\end{enumerate}

The problem takes the form:
\begin{center} \begin{tikzpicture}
	\node[ellipse, draw] (0) {$S_0$};
	\node[ellipse, draw, right =of 0] (1) {$S_1$};
	\node[right =of 1] (2) {$...$};
	\node[ellipse, draw, right =of 2] (f) {$S_f$};
	\draw [->] (0.east) -- node [align=center, above] {$a_0$} (1.west);
	\draw [->] (1.east) -- node [align=center, above] {$a_1$} (2.west);
	\draw [->] (2.east) -- node [align=center, above] {$a_{f-1}$} (f.west);
\end{tikzpicture} \end{center}

with the solution \begin{equation*} [a0, a1, … , a_{n-1}] \end{equation*}
\\
We will introduce a few “toy problems” to learn to answer these:\\

8 PUZZLE\\

\begin{center} \begin{tikzpicture}
	\node (1) {\begin{tabular} { | c | c | c | } 
		\hline 5 & 4 &  \\ \hline 6 & 1 & 8 \\ \hline 7 & 3 & 2 \\\hline\end{tabular}};
	\node[right =of 1] (2) {\begin{tabular} { | c | c | c | } 
		\hline 5 &  & 4 \\ \hline 6 & 1 & 8 \\ \hline 7 & 3 & 2 \\\hline\end{tabular}};
	\node [right =of 2] (3) {...};
	\node[right =of 3] (4) {\begin{tabular} { | c | c | c | } 
		\hline 1 & 2 & 3 \\ \hline 8 &  & 4 \\ \hline 7 & 5 & 6 \\\hline\end{tabular}};
	\draw [->] (1.east) -- (2.west);
	\draw [->] (2.east) -- (3.west);
	\draw [->] (3.east) -- (4.west);
\end{tikzpicture} \end{center}

\noindent Let’s work through this; it is tempting to say we can move any tile in one of 4 directions, \\
\indent BUT we need to simplify — that is too many choices.\\
Instead, we think about moving the empty space…now we have 4 choices max!

\noindent Let’s build a search tree for a 2x2 version of the problem:
\begin{itemize}[itemsep=0mm]
\item a path is a sequence of choices
\item each action has a cost (all 1 in this example)
\item min cost of this example is 3
\item since the tree is infinite, other paths work
\item dots represent dead-end repeated paths
\end{itemize} \medskip

\begin{center} \begin{tikzpicture}
	\node {\begin{tabular} { | c | c | } \hline & 1 \\ \hline 3 & 2 \\\hline\end{tabular}}
		child {node {\begin{tabular} { | c | c | } \hline 1 & \\ \hline 2 & 3 \\\hline\end{tabular}}
			child {node {\begin{tabular} { | c | c | } \hline 1 & 2 \\ \hline 3 &  \\\hline\end{tabular}}
				child {node {\begin{tabular} { | c | c | } \hline 1 & 2 \\ \hline & 3 \\\hline\end{tabular}}
					edge from parent node {}
				}
				child {node [ellipse, draw, fill] {}
					edge from parent node{}
				}
				edge from parent node {}
			}
			child {node [ellipse, draw, fill] {}
				edge from parent node{}
			}
		}
		child [missing]
		child {node {\begin{tabular} { | c | c | } \hline 3 & 1 \\ \hline  & 2 \\\hline\end{tabular}}
			child {node {\begin{tabular} { | c | c | } \hline 3 & 1 \\ \hline 2 &  \\\hline\end{tabular}}
				child {node {\begin{tabular} { | c | c | } \hline 3 & 2 \\ \hline 1 &  \\\hline\end{tabular}}
					edge from parent node {}
				}
				child {node [ellipse, draw, fill] {}
					edge from parent node{}
				}
				edge from parent node {}
			}
			child {node [ellipse, draw, fill] {}
				edge from parent node{}
			}
		};
\end{tikzpicture} \end{center}
\noindent When we perform a search like this, we can see that the tree growth exponentially with height.\\
What determines the complexity of a search problem?
\begin{enumerate}[itemsep=0mm]
\item branching factor (average choices)
\item number of possible states
\item depth of solution
\end{enumerate}

\noindent We were able to just walk through the last problem because it is so short; \\
	let’s consider a more complicated one:\\

\noindent MISSIONARIES AND CANNIBALS\\
\indent We have three missionaries and three cannibals with a boat on one side of a river.\\
\indent We wish to ferry everyone across, but the boat can only ferry up to two people at once.\\
\indent We cannot let there be more cannibals than missionaries in any location.\\
The third statement in this problem is called a constraint; any state violating this constraint is invalid.\\
\\
We will lay out the solution to this problem using LISP:
\begin{lstlisting} [language=Lisp]
; We must first answer the question: How do I represent a state?
> (M C B)
; M & C represent people on boat side, B is bool for (boat on right)
let initial-state = (3 3 t) 
let final-state = (3 3 NIL) 
; we define a final-function that checks for final-state equivalence
> (fin-fcn '(3 3 t))
> NIL
> (fin-fcn '(3 3 NIL))
> t
; How do I change states? We define a successor function:
> (succ-fn '(3 3 t))
> ((0 1 NIL)
    (1 1 NIL)
    (0 2 NIL))
; as we can see, only three of the five choices are valid
\end{lstlisting}
\noindent We must therefore pass three things to the function:
\begin{enumerate} [itemsep=0mm]
\item the initial state
\item the final-state function
\item the successor function
\end{enumerate}
We can now move into a slightly more abstract level of problem:\\
\subsection*{Constraint-Satisfaction Problems (CSP)}
\noindent These problems have no known solution on start, BUT we know the finality of the search tree!\\
\\
N QUEENS PROBLEM\\
\indent We wish to place N queens on an NxN chess board such that none can capture any other.\\
\indent We use the following definitions:\\
\begin{itemize} [itemsep=0mm]
\item STATE -- partially-filled board
\item INITIAL STATE -- empty board
\item FINALITY FUNCTION -- none of the queens are on the same row, column, or diagonal
\end{itemize}

\noindent There are many other forms of problem that can be solved by a blind search:
\begin{itemize}[itemsep=0mm]
	\item ROUTE FINDING
		\begin{itemize}[itemsep=0mm]
			\item geographic navigation
			\item computer network routing
			\item automated travel advisory systems
		\end{itemize}
	\item TRAVELING SALESMAN
		\begin{itemize}[itemsep=0mm]
			\item street-cleaning
			\item mail-delivery
		\end{itemize}
	\item VLSI LAYOUT (on chips)
		\begin{itemize}[itemsep=0mm]
			\item cell layout (group components into calls)
			\item channel routing (between cells)
		\end{itemize}
	\item ROBOT NAVIGATION
		\begin{itemize}[itemsep=0mm]
			\item continuous navigation on flat plane
			\item navigation of limbs and components
		\end{itemize}
	\item SCHEDULING
		\begin{itemize}[itemsep=0mm]
			\item automatic assembly sequencing
			\item protein design
		\end{itemize}
\end{itemize}


\end{document}